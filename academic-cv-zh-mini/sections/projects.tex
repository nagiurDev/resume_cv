\section{\Large 关键项目}
\vspace{0.3cm}

\begin{itemize}[leftmargin=0.0in]
    \setlength\itemsep{.2em}

    \cventry{}{ \link{https://github.com/nagiurDev/grammar-sentiment-emotion-analysis} }{Fine-tuning Pre-trained for Writing Improvement}{\textbf{NLP, transformer}}{}{ \setlength{\itemindent}{.2in} \setlength\itemsep{.3em}
    \vspace{5pt}
    \item \textbf{数据集:} CoLA, Lang-8, SST-2, GoEmotions
    \item \textbf{模型:} RoBERTa (base), FLAN-T5 (base)
    \item \textbf{挑战:} 优化多目标(可能存在冲突)的流水线。
    \item \textbf{主要成果 :} 在多任务学习中平衡任务特定优化。
    }


    \cventry{}{\link{https://github.com/nagiurDev/news-classification}}{News Classification}{\textbf{NLP, Web Scraping, Data Preprocessing, ML}}{}{ \setlength{\itemindent}{.2in} \setlength\itemsep{.3em}
    \vspace{5pt}
    \item 从 People.cn抓取新闻数据。
    \item 清洗、预处理和探索新闻数据。
    \item 构建和评估用于新闻分类的支持向量机(SVM)模型。
    \item 使用 Jupyter Notebook进行数据处理和模型构建。
    }




    \vspace{0.4cm}
    \cventry{}{\link{https://github.com/nagiurDev/sst2-sentiment-analysis-comparison}}{SST-2 Sentiment Analysis Comparison}{\textbf{NLP, Fine-tuning, Benchmarking, ML}}{}{ \setlength{\itemindent}{.2in} \setlength\itemsep{.3em}
    \vspace{5pt}
    \item 对比BERT、DistilBERT和RoBERTa在情感分析中的性能。
    \item 使用SST-2数据集进行基准测试。
    \item 使用准确率和F1值评估模型性能。
    \item 提供代码和结果,确保结果的可重复性。
    }

    
\end{itemize}


